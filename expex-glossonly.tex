% !TEX TS-program = pdflatexmk
\def\EPGLversion{0.5}
\def\EPGLdate{July 31, 2016}
\documentclass[11pt]{article}
\title{\textbf{The \textsf{expex-glossonly} package}}
\author{\textbf{Alan Munn}\\Department of Linguistics and Languages\\Michigan State University\\\texttt{\href{mailto:amunn@msu.edu}{amunn@msu.edu}}}
\date{Version \EPGLversion\\\EPGLdate}
\usepackage[T1]{fontenc}
\usepackage[margin=1in,includefoot]{geometry}
\usepackage{titling}
\usepackage[utf8]{inputenc}
\usepackage{array, booktabs, multicol, fancyhdr, xspace,tabularx}
\usepackage{enumitem}
\usepackage{fancyvrb,listings,url}
\usepackage[sf,compact]{titlesec}
\usepackage[]{he-she}
\usepackage{gb4e}
\usepackage[colorlinks=true]{hyperref}
\noautomath


\DefineShortVerb{\|}
\newcommand*\bs{\textbackslash}

\IfFileExists{luximono.sty}%
{%
  \usepackage[scaled]{luximono}%
}
{%
  \IfFileExists{beramono.sty}%
  {%
    \usepackage[scaled]{beramono}%
  }{}
}

  
\lstset{%
    basicstyle=\ttfamily\small,
    commentstyle=\itshape\ttfamily\small,
    showspaces=false,
    showstringspaces=false,
    breaklines=true,
    breakautoindent=true,
    breakindent=1em,
    xrightmargin=2.5em,
    captionpos=t
    language=TeX
}
  
\newcommand*{\pkg}[1]{\texttt{#1}\xspace}
\setitemize[1]{label={}}
\setitemize[2]{label={}}
\setdescription{font={\normalfont}}
\setlength{\droptitle}{-1in}

\lhead{}
\chead{}
\rhead{}
\lfoot{\emph{}}
\cfoot{\thepage}
\rfoot{}
\renewcommand{\headrulewidth}{0pt}
\renewcommand{\footrulewidth}{0pt}
\pagestyle{fancy}


\begin{document}
\maketitle
\thispagestyle{empty}
\renewcommand{\abstractname}{\sffamily Abstract}
\abstract{\noindent\begin{quote}This is an experimental package which is designed to let \pkg{gb4e} and \pkg{linguex} package users use the advanced glossing capabilities of the very powerful ExPex package.\end{quote}}
\section{Introduction}
The ExPex package by John Frampton provides very fine-grained control over glossing and example formatting, including unlimited gloss lines and various ways of formatting multiline glosses.  By contrast the \pkg{cgloss4e} glossing macros provided with both \pkg{gb4e} and \pkg{linguex}, although very capable at basic glossing, lack the degree of customization that is sometimes needed for more complex glossing.

On the other hand, for those users who have heavily invested in using either \pkg{gb4e} or \pkg{linguex}, shifting to ExPex can be quite daunting and burdensome, especially since the basic syntax of the examples is quite different.

This package is an attempt to have the best of both worlds: it allows \pkg{gb4e} and \pkg{linguex} users to keep using those packages for basic example numbering and formatting, but also allows them to use the glossing macros that ExPex provides.

\section{Package usage}
Usage of the package is simple: simply load it, and then load your favourite example numbering package. The \pkg{expex-glossonly} package must be loaded first.

\begin{multicols}{2}
\begin{lstlisting}
\usepackage{expex-glossonly}
\usepackage{gb4e}
\end{lstlisting}
\begin{lstlisting}
\usepackage{expex-glossonly}
\usepackage{linguex}
\end{lstlisting}
\end{multicols}



The package checks for which example numbering package you have loaded (either \pkg{gb4e} or \pkg{linguex}) and then patches the main glossing macro in ExPex to adjust to the horizontal spacing parameters of the particular example package you loaded.

It does nothing explicitly to disable incompatible commands from ExPex (specifically any commands that would introduce a numbered or lettered example, such as |\ex|, |\pex|, |\a|, etc.) so you should not use them. Some, such as |\ex| are redefined by \pkg{gb4e} and therefore the ExPex version of it will fail to work.  For this reason, you need to make sure to load the \pkg{expex-glossonly} package \emph{before} you load \pkg{gb4e} or \pkg{linguex}
\section{Bugs}
This is \emph{experimental} and has not been extensively tested. Use at your own risk. You're welcome to contact me with problems as they arise, however.
\end{document}